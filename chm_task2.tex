\documentclass [a4paper, 12pt] {article}
\usepackage [russian] {babel}
\usepackage [utf8] {inputenc}
\usepackage {amsmath}
\usepackage {amssymb}
\usepackage {geometry} % Меняем поля страницы
\geometry {left=1cm}% левое поле
\geometry {right=1cm}% правое поле
\geometry {top=1cm}% верхнее поле
\geometry {bottom=2cm}% нижнее поле
\setcounter{page} {1}
\usepackage{graphicx}
\graphicspath{{images/}}

\begin {document}

\begin {center}
Отчёт по теме "Численное решение задачи оптимального управления"
\end {center}

Группа: 410

Студент: Лапин Александр

\section {Постановка задачи}

Требуется найти численное решение следующей задачи оптимального управления

$$\int_0^{\frac{\pi}{2}} u^2 \, dt + x^2 (0) \rightarrow inf,$$
$$\dot x(0) = x(\pi / 2) = 0, \dot x(\pi / 2) = 1, \ddot x + x \exp (- \alpha x) = u, \alpha \in [0.0, 25.0]$$
\section {Алгоритм решения}

\subsection {Сведение задачи к системе ОДУ с краевыми условиями}

После сведения задачи к системе ОДУ с краевыми условиями, получаем следующую систему:

$$
\dot x_1 = x_2
$$
$$
\dot x_2 = u - x_1 e^{- \alpha x_1}
$$
$$
\dot p_1 = e^{- \alpha x_1} (1 - \alpha x_1) \cdot p_2
$$
$$
\dot p_2 = - p_1
$$
$$
p_1(0) = x_1(0)
$$
$$
p_2 = u
$$
$$
x_2 (0) = 0
$$
$$
x_1(\frac{\pi}{2}) = 0
$$
$$
x_2(\frac{\pi}{2}) = 1
$$

\subsection {Алгорим численного решения полученной системы}
Неизвестных функций в системе 4 штуки, при этом на концах известно только два краевых условия из 4-х. Следовательно надо найти два недостающих краевых условия. Для определённости искать их будем в начале отрезка в 0. Итак, пусть:

$$
\left
\{ 
\begin {array} {ll}
p_1(0) = a_1, \\
p_2(0) = a_2. \\
\end {array}
\right .
$$

Найдём параметры $a_1$ и $a_2$ с помощью модифицированного алгоритма Ньютона с использованием нормировки Федоренко. Для решения нам нужны следующие параметры:

\begin{enumerate}

\item  $\epsilon$ --- требуемая точность решения.

\item $\delta$ --- приращение аргумента для поиска производной.

\item $\tau$ --- длина шага в алгоритме Рунге-Кутты.

\end{enumerate}

Алгоритм выполняется в несколько этапов. Пусть номер итерации задаёт чисто $k$.

\begin {enumerate}

\item $k = 0$
Возьмём за начальное приближение параметров $a_1$ и $a_2$ данные, полученные при подстановке $T = 2\pi$. Получаем начальное приближение $a_1^0 = 1, a_2^0 = 0$. С помощью алгоритма Рунге-Кутта, использовав наше приближение, найдём $c_1^0 = x_1(T)$ и $c_2^0 = x_2(T)$. Получаем вектор невязки:

$$X^0 = (c_1^0, c_2^0 - 1).$$

Для всех итераций вектор невязки выглядит так:

$$X^k = (c_1^k, c_2^k - 1).$$

Посчитаем невязку:
$S^0 = ||X||_2 = \sqrt{(X_1^0)^2 + (X_2^0)}$

Если $S^0 < \epsilon$, то заканчиваем выполнение алгоритма, так как мы нашли подходящие краевые условия. Если нет, переходим к следующей итерации алгоритма.

\item $k > 0$

Пусть текущее приближение краевых условий $a^k = (a_1^k, a_2^k)$, вектор невязки $X^k = (c_1^k, c_2^k - 1)$, $\gamma_k = 1$. Найдём следующее приближение краевых условий с помощью реккурентной формулы:
$$a^{k+1} = (a^k)^T - \gamma_k (\dot X^k)^{-1}(X^k)^T,$$

где $(\dot X^k)$ --- матрица Якоби, частные производные ищутся по формуле:

$$\genfrac{}{}{}{}{dX_i^k}{da_j^k} = \genfrac{}{}{}{}{X_i^k(a_1^k, ..., a_j^k + \delta, ...) - X_i^k(a_1^k, ..., a_j^k, ...)}{\delta}$$

Дальше считаем невязку с нормой Федоренко. Нормировочные множители такие:

$$k_1^2 = j_{11}^2 + j_{12}^2,$$
$$k_2^2 = j_{21}^2 + j_{22}^2,$$

где $(j_{il})$ - только что найденная матрица Якоби. После этого считаем норму

$$||X^{k+1}||_{\Phi} = \sqrt{(\genfrac{}{}{}{}{X_1^{k+1}}{k_1})^2 + (\genfrac{}{}{}{}{X_2^{k+1}}{k_2})^2}$$

Если $||X^{k+1}||_{\Phi} < \epsilon$, то заканчиваем алгоритм, ответ мы нашли. Если  $||X^{k+1}||_{\Phi} < X^{k}||_{\Phi}$, то переходим к следующей итерации алгоритма.  Если  $||X^{k+1}||_{\Phi} \ge X^{k}||_{\Phi}$, то уменьшаем $\gamma_k$ в двое и пересчитываем всю итерацию.
Начальные приближения находились перебором.

\end {enumerate}

\subsection {Поиск значения функционала}

Значение функционала будем искать с помощью формулы Симпсона. $h\_simp$ --- шаг разбиения итеграла на сумму интегралов:

$$\int_0^{\pi / 2} u^2 \, dt = \sum_k\int_{x_k}^{x_{k+1}} u^2\, dt$$

$$\int_{x_k}^{x_{k+1}} u^2 \, dt \approx \genfrac{}{}{}{}{h\_simp}{6}( u^2(x_k) + 4  u^2(\genfrac{}{}{}{}{x_k + x_{k+1}}{2}) +  u^2(x_{k+1}))$$

\section {Численные результаты}

Во всех приведённых в таблице результатах $\epsilon < 1e-7, \delta = 1e-5$

\begin{table}[h!]
\begin{center}
\begin{tabular}{llllll}
$\alpha$  & $\tau$ & h\_sim & $p_1(0)$   & $p_2(0)$   & result \\
0   & 1e-2    & 1e-2    & -0.6816222           & -0.4339341          & 0.21701335789   \\
5   & 1e-2    & 1e-2    & -0.0550866 & -1.224772669   & 1.051751810209 \\
10 & 1e-2    & 1e-2    & 0.14338431 & -0.727341818 & 0.7684414831 \\
15 & 1e-2    & 1e-2    & 0.2333231563 & -0.5793537491 & 0.7335585168 \\
20 & 1e-2    & 1e-2    & 0.2932859841   & -0.5102745366   &  0.723388841\\
25 & 1e-2    & 1e-2    & 0.334583262  & -0.461841896   &  0.71827762311 \\
\end{tabular}
\end{center}
\end{table}

\end {document}